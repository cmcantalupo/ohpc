Global Extensible Open Power Manager (\GEOPM{}) is an extensible power
management framework targeting high performance computing.  The library can be
extended to support new control algorithms and new hardware power management
features.  The \GEOPM{} package provides built in features ranging from static
management of power policy for each individual compute node, to dynamic
coordination of power policy and performance across all of the compute nodes
hosting one MPI job on a portion of a distributed computing system.  The
dynamic coordination is implemented as a hierarchical control system for
scalable communication and decentralized control.  The following
commands can be used to enable development on the {\em master} host and
execution of the \GEOPM{} runtime on the {\em compute} hosts.

% begin_ohpc_run
% ohpc_validation_newline
% ohpc_command if [[ ${enable_geopm} -eq 1 ]];then
% ohpc_indent 5
% ohpc_validation_comment Install GEOPM on master and compute nodes
\begin{lstlisting}[language=bash,keywords={},upquote=true]
# Install GEOPM package on master
[sms](*\#*) (*\install*) geopm-ohpc

# Install GEOPM package on compute node
[sms](*\#*) (*\chrootinstall*) geopm-ohpc
\end{lstlisting}
% ohpc_indent 0
% ohpc_command fi
% end_ohpc_run

\noindent Installing the \GEOPM{} package on the {\em master} host will enable
users to compile against the \GEOPM{} libraries and include headers, and
installing on the {\em compute} hosts will enable applications to use
\GEOPM{} at runtime.  For documentation on how to use \GEOPM{} please see
the \GEOPM{} man pages which are all linked from the geopm(7) overview
man page available in html here:
\href{http://geopm.github.io/man/geopm.7.html}{\color{blue}{http://geopm.github.io/man/geopm.7.html}}.
Please see the \GEOPM{} tutorials and for working examples using the
\GEOPM{} runtime here:
\href{https://github.com/geopm/geopm/tree/dev/tutorial}{\color{blue}{https://github.com/geopm/geopm/tree/dev/tutorial}}.
