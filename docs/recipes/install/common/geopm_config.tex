The Global Extensible Open Power Manager (\GEOPM{}) is a framework for
exploring power and energy optimizations targeting high performance computing.
The \GEOPM{} package provides built-in features ranging from static management
of power policy for each individual compute node, to dynamic coordination of
the power policy and performance across all compute nodes hosting an MPI
application on a portion of a distributed computing system.  The dynamic
coordination is implemented as a hierarchical control system for scalable
communication and decentralized control. The following commands customize the
provisioning environment to support \GEOPM{} installation which is done in a
later step in \S\ref{sec:install_perf_tools}.

% begin_ohpc_run
% ohpc_validation_newline
% ohpc_validation_comment Optionally, update compute image to support geopm
% ohpc_command if [[ ${enable_geopm} -eq 1 ]];then
% ohpc_indent 5
\begin{lstlisting}[language=bash,keywords={},upquote=true]
# Disable Intel pstate driver for compute nodes as it interferes with GEOPM's operation.
[sms](*\#*) export kargs="${kargs} intel_pstate=disable"
\end{lstlisting}
% ohpc_indent 0
% ohpc_command fi
% end_ohpc_run

\noindent \GEOPM{} uses the \texttt{msr-safe} kernel module
to allow users read/write access to whitelisted model specific
registers (MSRs).  An associated \SLURM{} plugin ensures that MSRs modified
within a user's slurm job are reset to their original state after job completion.
%before the compute node is returned to the pool available to other
%users of the system.

% begin_ohpc_run
% ohpc_validation_newline
% ohpc_command if [[ ${enable_geopm} -eq 1 ]];then
% ohpc_indent 5
\begin{lstlisting}[language=bash,keywords={},upquote=true]
# Install msr-safe kernel module, supporting scripts, and SLURM plugin
[sms](*\#*) (*\chrootinstall*) kmod-msr-safe-ohpc
[sms](*\#*) (*\chrootinstall*) msr-safe-ohpc
[sms](*\#*) (*\chrootinstall*) msr-safe-slurm-ohpc
\end{lstlisting}
% ohpc_indent 0
% ohpc_command fi
% end_ohpc_run

\noindent For documentation on how to configure and use \GEOPM{}, please see
the \texttt{geopm} man page and tutorials
available \href{https://github.com/geopm/geopm/tree/dev/tutorial}{online}.

