\documentclass[letterpaper]{article}
\usepackage{common/ohpc-doc}
\setcounter{secnumdepth}{5}
\setcounter{tocdepth}{5}

% Include git variables
\input{vc.tex}

% Define Base OS and other local macros
\newcommand{\baseOS}{CentOS7.5}
\newcommand{\OSRepo}{CentOS\_7.5}
\newcommand{\OSTree}{CentOS\_7}
\newcommand{\OSTag}{el7}
\newcommand{\baseos}{centos7.5}
\newcommand{\baseosshort}{centos7}
\newcommand{\provisioner}{xCAT}
\newcommand{\provheader}{xCAT (stateful)}
\newcommand{\rms}{SLURM}
\newcommand{\rmsshort}{slurm}
\newcommand{\arch}{x86\_64}
\newcommand{\installimage}{install}
%%% WARNING: Hack below. The version should be read from ohpc-doc.sty, but the
%%% perl parsing script does not read that file. This works for one release, but
%%% needs a proper fix.
\newcommand{\VERLONG}{1.3.6}

% Define package manager commands
\newcommand{\pkgmgr}{yum}
\newcommand{\addrepo}{wget -P /etc/yum.repos.d}
\newcommand{\chrootaddrepo}{wget -P \$CHROOT/etc/yum.repos.d}
\newcommand{\clean}{yum clean expire-cache}
\newcommand{\chrootclean}{yum --installroot=\$CHROOT clean expire-cache}
\newcommand{\install}{yum -y install}
\newcommand{\chrootinstall}{psh compute yum -y install}
\newcommand{\groupinstall}{yum -y groupinstall}
\newcommand{\groupchrootinstall}{psh compute yum -y groupinstall}
\newcommand{\remove}{yum -y remove}
\newcommand{\upgrade}{yum -y upgrade}
\newcommand{\chrootupgrade}{yum -y --installroot=\$CHROOT upgrade}
\newcommand{\tftppkg}{syslinux-tftpboot}
\newcommand{\beegfsrepo}{https://www.beegfs.io/release/beegfs\_6/dists/beegfs-rhel7.repo}

% boolean for os-specific formatting
\toggletrue{isCentOS}
\toggletrue{isCentOS_ww_slurm_x86}
\toggletrue{isSLURM}
\toggletrue{isx86}
\toggletrue{isxCAT}
\toggletrue{isxCATstateful}

\begin{document}
\graphicspath{{common/figures/}}
\thispagestyle{empty}

% Title Page
\input{common/title}
% Disclaimer 
\newpage

\vspace*{3.0cm}
\noindent {\Large \color{logoblue} \fontfamily{phv}\selectfont Legal Notice} \\ 

\vspace*{0.5cm}

\noindent Copyright {\small\copyright} 2016-2020, OpenHPC, a Linux Foundation
Collaborative Project. All rights reserved. \\

\vspace*{0.1cm}

\noindent \begin{tabular}{cp{10cm}}
\raisebox{-.75\height}{\includegraphics[width=0.22\textwidth]{cc_by}} &
This documentation is licensed under the Creative Commons Attribution 4.0 International
License. To view a copy of this license, visit
\href{http://creativecommons.org/licenses/by/4.0}{\color{blue}{http://creativecommons.org/licenses/by/4.0}}. \\
\end{tabular}


\vspace*{1.5cm}

{\footnotesize

\noindent Intel, the Intel logo, and other Intel marks are trademarks of Intel
Corporation in the U.S. and/or other countries. \\
\iftoggleverb{ispbs}
\noindent Altair, the Altair logo, PBS Professional, and other Altair marks are
trademarks of Altair Engineering, Inc. in the U.S. and/or other countries. \\
\fi
\noindent *Other names and brands may be claimed as the property of others. \\



}
 

\newpage
\tableofcontents
\newpage

% Introduction  --------------------------------------------------

\section{Introduction} \label{sec:introduction}
\input{common/install_header}
\input{common/intro} \\

\input{common/base_edition/edition}
\subsection{Target Audience}

This guide is targeted at experienced \Linux{} system administrators for HPC
environments. Knowledge of software package management, system networking, and
PXE booting is assumed. Command-line input examples are highlighted throughout
this guide via the following syntax:

\begin{lstlisting}[language=bash,literate={-}{-}1,keywords={},upquote=true]
[sms](*\#*) echo "OpenHPC hello world"
\end{lstlisting}

Unless specified otherwise, the examples presented are executed with
elevated (root) privileges. The examples also presume use of the BASH login
shell, though the equivalent commands in other shells can be substituted.
In addition to specific command-line instructions called out in this guide, an
alternate convention is used to highlight potentially useful tips or optional
configuration options. These tips are highlighted via the following format:

\begin{center}
\begin{tcolorbox}[]
\small  Training is everything. The peach was once a bitter almond; cauliflower is nothing but cabbage with a college education. --Mark Twain
\end{tcolorbox}
\end{center}


\input{common/requirements}
\input{common/inputs}


% Base Operating System --------------------------------------------

\section{Install Base Operating System (BOS)}
\input{common/bos}

%\clearpage 
% begin_ohpc_run
% ohpc_validation_newline
% ohpc_validation_comment Disable firewall 
\begin{lstlisting}[language=bash,keywords={}]
[sms](*\#*) systemctl disable firewalld
[sms](*\#*) systemctl stop firewalld
\end{lstlisting}
% end_ohpc_run

% ------------------------------------------------------------------

\section{Install \xCAT{} and Provision Nodes with BOS} \label{sec:provision_compute_bos}
\input{common/xcat_stateful_compute_bos_intro}

\subsection{Enable \xCAT{} repository for local use} \label{sec:enable_xcat}
\input{common/enable_xcat_repo}

\noindent \xCAT{} has a number of dependencies that are required for
installation that are housed in separate public repositories for various
distributions. To enable for local use, issue the following:

% begin_ohpc_run
\begin{lstlisting}[language=bash,keywords={},basicstyle=\fontencoding{T1}\fontsize{8.0}{10}\ttfamily,literate={ARCH}{\arch{}}1 {-}{-}1]
[sms](*\#*) (*\addrepo*) https://xcat.org/files/xcat/repos/yum/xcat-dep/rh7/ARCH/xcat-dep.repo
\end{lstlisting}
% end_ohpc_run

\subsection{Add provisioning services on {\em master} node} \label{sec:add_provisioning}
\input{common/install_provisioning_xcat_intro_stateful}
%\input{common/enable_pxe}

\vspace*{-0.15cm}
\subsection{Complete basic \xCAT{} setup for {\em master} node} \label{sec:setup_xcat}
\input{common/xcat_setup}


\subsection{Define {\em compute} image for provisioning}
\input{common/xcat_init_os_images_centos}

\subsection{Add compute nodes into \xCAT{} database} \label{sec:xcat_add_nodes}
\input{common/add_xcat_hosts_intro}

%\vspace*{-0.25cm}
\subsection{Boot compute nodes} \label{sec:boot_computes}
\input{common/reset_computes_xcat} 




\section{Install \OHPC{} Components} \label{sec:basic_install}
\input{common/install_ohpc_components_intro}


\subsection{Enable \OHPC{} repository for local use} \label{sec:enable_repo}
\input{common/enable_local_ohpc_repo}

% begin_ohpc_run
% ohpc_validation_newline
% ohpc_validation_comment Verify OpenHPC repository has been enabled before proceeding
% ohpc_validation_newline
% ohpc_command yum repolist | grep -q OpenHPC
% ohpc_command if [ $? -ne 0 ];then
% ohpc_command    echo "Error: OpenHPC repository must be enabled locally"
% ohpc_command    exit 1
% ohpc_command fi
% end_ohpc_run


In addition to the \OHPC{} and \xCAT{} package repositories, the {\em master} host also
requires access to the standard base OS distro repositories in order to resolve
necessary dependencies. For \baseOS{}, the requirements are to have access to
both the base OS and EPEL repositories for which mirrors are freely available online:

\begin{itemize*}
\item CentOS-7 - Base 7.5.1804
  (e.g. \href{http://mirror.centos.org/centos-7/7/os/x86\_64}
             {\color{blue}{http://mirror.centos.org/centos-7/7/os/x86\_64}} )
\item EPEL 7 (e.g. \href{http://download.fedoraproject.org/pub/epel/7/x86\_64}
                        {\color{blue}{http://download.fedoraproject.org/pub/epel/7/x86\_64}} )
\end{itemize*}

\noindent The public EPEL repository is enabled by installing
\texttt{epel-release} package. Note that this requires the CentOS Extras
repository, which is shipped with CentOS and is enabled by default.

% begin_ohpc_run
\begin{lstlisting}[language=bash,keywords={},basicstyle=\fontencoding{T1}\fontsize{8.0}{10}\ttfamily,literate={ARCH}{\arch{}}1 {-}{-}1]
[sms](*\#*)  (*\install*) epel-release
\end{lstlisting}
% end_ohpc_run

Now \OHPC{} packages can be installed. To add the base package on the SMS
issue the following
% begin_ohpc_run
\begin{lstlisting}[language=bash,keywords={},basicstyle=\fontencoding{T1}\fontsize{8.0}{10}\ttfamily,literate={ARCH}{\arch{}}1 {-}{-}1]
[sms](*\#*)  (*\install*) ohpc-base
\end{lstlisting}
% end_ohpc_run


\input{common/automation}


\subsection{Setup time synchronization service on {\em master} node} \label{sec:add_ntp}
HPC systems rely on synchronized clocks throughout the system and the
NTP protocol can be used to facilitate this synchronization. To enable NTP
services on the SMS host with a specific server \texttt{\$\{ntp\_server\}},
issue the following:

% begin_ohpc_run
% ohpc_validation_comment Enable NTP services on SMS host
\begin{lstlisting}[language=bash,literate={-}{-}1,keywords={},upquote=true,keepspaces]
[sms](*\#*) systemctl enable chronyd.service
[sms](*\#*) echo "server ${ntp_server}" >> /etc/ntp.conf
[sms](*\#*) systemctl restart ntpd
\end{lstlisting}
% end_ohpc_run


\subsection{Add resource management services on {\em master} node} \label{sec:add_rm}
\input{common/install_slurm}

\subsection{Optionally add \InfiniBand{} support services on {\em master} node} \label{sec:add_ofed}
\input{common/ibsupport_sms_centos}

\subsection{Optionally add \OmniPath{} support services on {\em master} node} \label{sec:add_opa}
\input{common/opasupport_sms_centos}

\vspace*{-0.2cm}
\subsubsection{Add \OHPC{} components to {\em compute} nodes} \label{sec:add_components}
\input{common/add_to_compute_stateful_xcat_intro}

%\newpage
% begin_ohpc_run
% ohpc_validation_comment Add OpenHPC components to compute instance
\begin{lstlisting}[language=bash,literate={-}{-}1,keywords={},upquote=true]
# Add Slurm client support meta-package
[sms](*\#*) (*\chrootinstall*) ohpc-slurm-client

# Add Network Time Protocol (NTP) support
[sms](*\#*) (*\chrootinstall*) ntp

# Add kernel drivers
[sms](*\#*) (*\chrootinstall*) kernel

# Include modules user environment
[sms](*\#*) (*\chrootinstall*) lmod-ohpc
\end{lstlisting}
% end_ohpc_run

% ohpc_comment_header Optionally add InfiniBand support services in compute node image \ref{sec:add_components}
% ohpc_command if [[ ${enable_ib} -eq 1 ]];then
% ohpc_indent 5
\begin{lstlisting}[language=bash,literate={-}{-}1,keywords={},upquote=true]
# Optionally add IB support and enable
[sms](*\#*) (*\groupchrootinstall*) "InfiniBand Support"
[sms](*\#*) (*\chrootinstall*) infinipath-psm
[sms](*\#*) psh compute systemctl enable rdma
[sms](*\#*) psh compute systemctl start rdma
\end{lstlisting}
% ohpc_indent 0
% ohpc_command fi
% end_ohpc_run

\vspace*{-0.25cm}
\subsubsection{Customize system configuration} \label{sec:master_customization}
\input{common/xcat_stateful_customize_centos}

% Additional commands when additional computes are requested

% begin_ohpc_run
% ohpc_validation_newline
% ohpc_validation_comment Update basic slurm configuration if additional computes defined
% ohpc_validation_comment This is performed on the SMS, nodes will pick it up config file is copied there later
% ohpc_command if [ ${num_computes} -gt 4 ];then
% ohpc_command    perl -pi -e "s/^NodeName=(\S+)/NodeName=${compute_prefix}[1-${num_computes}]/" /etc/slurm/slurm.conf
% ohpc_command    perl -pi -e "s/^PartitionName=normal Nodes=(\S+)/PartitionName=normal Nodes=${compute_prefix}[1-${num_computes}]/" /etc/slurm/slurm.conf
% ohpc_command fi
% end_ohpc_run

%\clearpage
\subsubsection{Additional Customization ({\em optional})} \label{sec:addl_customizations}
This section highlights common additional customizations that can {\em
optionally} be applied to the local cluster environment. These customizations
include:

\begin{multicols}{2}
\begin{itemize*}
\iftoggleverb{isx86}
\item Add InfiniBand or Omni-Path drivers
\item Increase memlock limits
\fi

\nottoggle{ispbs}{\item Restrict ssh access to compute resources}

\iftoggleverb{isx86}
\item Add \beegfs{} client
\item Add \Lustre{} client
\fi

\iftoggle{isWarewulf}{\item Enable syslog forwarding}

\item Add \Nagios{} Core monitoring
\item Add \Ganglia{} monitoring
\item Add \Sensys{} monitoring
\item Add \clustershell{}
\item Add \mrsh{}
\item Add \genders{}
%%\item Add \powerman{}
\item Add \conman{}  
\item Add \GEOPM{}
\end{itemize*}
\end{multicols}

\noindent Details on the steps required for each of these customizations are
discussed further in the following sections.


\paragraph{Increase locked memory limits}
\input{common/memlimits_stateful}

\paragraph{Enable ssh control via resource manager} 
\input{common/slurm_pam_stateful}

\paragraph{Add \Lustre{} client} \label{sec:lustre_client}
\input{common/lustre-client}
\input{common/lustre-client-centos-stateful}
\input{common/lustre-client-post-stateful}

\paragraph{Add \Nagios{} monitoring}
\input{common/nagios_stateful}

\vspace*{0.4cm}
\paragraph{Add \Ganglia{} monitoring}
\input{common/ganglia_stateful}

\paragraph{Add \clustershell{}}
\clustershell{} is an event-based Python library to execute commands in parallel
across cluster nodes. Installation and basic configuration defining three node
groups ({\em adm}, {\em compute}, and {\em all}) is as follows:

% begin_ohpc_run
% ohpc_validation_newline
% ohpc_command if [[ ${enable_clustershell} -eq 1 ]];then
% ohpc_indent 5
% ohpc_validation_comment Install clustershell
\begin{lstlisting}[language=bash,keywords={},upquote=true]
# Install ClusterShell
[sms](*\#*) (*\install*) clustershell

# Setup node definitions
[sms](*\#*) cd /etc/clustershell/groups.d
[sms](*\#*) mv local.cfg local.cfg.orig
[sms](*\#*) echo "adm: ${sms_name}" > local.cfg
[sms](*\#*) echo "compute: ${compute_prefix}[1-${num_computes}]" >> local.cfg
[sms](*\#*) echo "all: @adm,@compute" >> local.cfg
\end{lstlisting}
% ohpc_indent 0
% ohpc_command fi
% end_ohpc_run



\paragraph{Add \mrsh{}}
\input{common/mrsh_stateful}

\paragraph{Add \genders{}}
\input{common/genders}

\clearpage
\paragraph{Add \conman{}} \label{sec:add_conman}
\input{common/conman}

\paragraph{Add \nhc{}} \label{sec:add_nhc}
Resource managers often provide for a periodic "node health check" to be
performed on each compute node to verify that the node is working
properly. Nodes which are determined to be "unhealthy" can be marked as down or
offline so as to prevent jobs from being scheduled or run on them. This helps
increase the reliability and throughput of a cluster by reducing preventable
job failures due to misconfiguration, hardware failure, etc. OpenHPC
distributes \nhc{} to fulfill this requirement.

In a typical scenario, the \nhc{} driver script is run periodically on each compute
node by the resource manager client daemon. It loads its
configuration file to determine which checks are to be run on the current node
(based on its hostname). Each matching check is run, and if a failure is
encountered, \nhc{} will exit with an error message describing the problem. It can
also be configured to mark nodes offline so that the scheduler will not assign
jobs to bad nodes, reducing the risk of system-induced job failures. 

% begin_ohpc_run
% ohpc_validation_newline
% ohpc_validation_comment Optionally, enable nhc and configure
\begin{lstlisting}[language=bash,keywords={},upquote=true]
# Install NHC on master and compute nodes
[sms](*\#*) (*\install*) nhc-ohpc
[sms](*\#*) (*\chrootinstall*) nhc-ohpc
\end{lstlisting}
% end_ohpc_run


% begin_ohpc_run
% ohpc_validation_newline
\begin{lstlisting}[language=bash,keywords={},upquote=true]
# Register as SLURM's health check program
[sms](*\#*) echo "HealthCheckProgram=/usr/sbin/nhc" >> /etc/slurm/slurm.conf
[sms](*\#*) echo "HealthCheckInterval=300" >> /etc/slurm/slurm.conf  # execute every five minutes
\end{lstlisting}
% end_ohpc_run




%\subsubsection{Identify files for synchronization} \label{sec:file_import}
%\input{common/import_xcat_files}
%\input{common/import_xcat_files_slurm}

%%%\subsubsection{Optional kernel arguments} \label{sec:optional_kargs}
%%%If you chose to enable \conman{} in \S\ref{sec:add_conman}, additional
warewulf configuration is needed as follows:
% begin_ohpc_run
% ohpc_validation_newline
% ohpc_validation_comment Optionally, enable console redirection
% ohpc_command if [[ ${enable_ipmisol} -eq 1 ]];then
% ohpc_indent 5
\begin{lstlisting}[language=bash,keywords={},upquote=true]
# Define node kernel arguments to support SOL console
[sms](*\#*) wwsh -y provision set "${compute_regex}" --console=ttyS1,115200
\end{lstlisting}
% ohpc_indent 0
% ohpc_command fi
% end_ohpc_run

%%% If any components have added to the boot time kernel command line argumenst for the compute nodes,
%%% the following command is required to store the configuration in Warewulf:
%%% % ohpc_validation_newline
%%% % ohpc_validation_comment Optionally, add arguments to bootstrap kernel
%%% % ohpc_command if [[ ${enable_kargs} -eq 1 ]]; then
%%% \begin{lstlisting}[language=bash,keywords={},upquote=true,basicstyle=\footnotesize\ttfamily]
%%% # Set optional compute node kernel command line arguments.
%%% [sms](*\#*) wwsh -y provision set "${compute_regex}" --kargs="${kargs}"
%%% \end{lstlisting}
%%% % ohpc_command fi


\section{Install \OHPC{} Development Components}
\input{common/dev_intro.tex}

%\vspace*{-0.15cm}
%\clearpage
\subsection{Development Tools} \label{sec:install_dev_tools}
\input{common/dev_tools}

\vspace*{-0.15cm}
\subsection{Compilers} \label{sec:install_compilers}
\OHPC{} presently packages the \GNU{} compiler toolchain integrated with the 
underlying modules-environment system in a hierarchical fashion. The modules
system will conditionally present compiler-dependent software based on the
toolchain currently loaded. 

% begin_ohpc_run
% ohpc_comment_header Install Compilers \ref{sec:install_compilers}
\begin{lstlisting}[language=bash]
[sms](*\#*) (*\install*) gnu9-compilers-ohpc
\end{lstlisting}
% end_ohpc_run

The llvm compiler toolchains are also provided as a standalone additional
compiler family (ie. users can easily switch between gcc/clang environments),
but we do not provide the full complement of downstream library builds.

% begin_ohpc_run
% ohpc_comment_header Install llvm Compilers
\begin{lstlisting}[language=bash]
[sms](*\#*) (*\install*) llvm5-compilers-ohpc
\end{lstlisting}
% end_ohpc_run


%\clearpage
\subsection{MPI Stacks} \label{sec:mpi}
For MPI development and runtime support, \OHPC{} provides pre-packaged builds
for a variety of MPI families and transport layers. Currently available options
and their applicability to various network transports are summarized in
Table~\ref{table:mpi}.  The command that follows installs a starting set of MPI
families that are compatible with ethernet fabrics. 

\iftoggleverb{isx86}
% x86_64

\begin{table}[h]
\caption{Available MPI variants} \label{table:mpi}
\centering
\begin{tabular}{@{\hspace*{0.2cm}} *5l @{}}    \toprule
                                  & Ethernet (TCP)                 & \InfiniBand{}                  & \IntelR{} Omni-Path            \\ \midrule
\rowcolor{black!10} MPICH         & \multicolumn{1}{c}{\checkmark} &                                &                                \\
MVAPICH2                          &                                & \multicolumn{1}{c}{\checkmark} &                                \\
\rowcolor{black!10} MVAPICH2 (psm2) &                              &                                & \multicolumn{1}{c}{\checkmark} \\
OpenMPI                           & \multicolumn{1}{c}{\checkmark} & \multicolumn{1}{c}{\checkmark} & \multicolumn{1}{c}{\checkmark} \\
\rowcolor{black!10} OpenMPI (PMIx) & \multicolumn{1}{c}{\checkmark} & \multicolumn{1}{c}{\checkmark} & \multicolumn{1}{c}{\checkmark} \\ \bottomrule
\end{tabular}
\end{table}

\else
% aarch64

\begin{table}[h]
\caption{Available MPI builds} \label{table:mpi}
\centering
\begin{tabular}{@{\hspace*{0.2cm}} *5l @{}}    \toprule
                                  & Ethernet (TCP)                 & \InfiniBand{}                              \\ \midrule
\rowcolor{black!10} MPICH         & \multicolumn{1}{c}{\checkmark} &                                            \\
\rowcolor{black!10} OpenMPI                           & \multicolumn{1}{c}{\checkmark} & \multicolumn{1}{c}{\checkmark} \\
\end{tabular}
\end{table}

\fi

% begin_ohpc_run
% ohpc_comment_header Install MPI Stacks \ref{sec:mpi}
% ohpc_command if [[ ${enable_mpi_defaults} -eq 1 && ${enable_pmix} -eq 0 ]];then
% ohpc_indent 5
\begin{lstlisting}[language=bash]
[sms](*\#*) (*\install*) openmpi4-gnu9-ohpc mpich-gnu9-ohpc
\end{lstlisting}
% ohpc_indent 0
% ohpc_command elif [[ ${enable_mpi_defaults} -eq 1 && ${enable_pmix} -eq 1 ]];then
% ohpc_indent 5
% ohpc_command (*\install*) openmpi4-pmix-slurm-gnu9-ohpc mpich-gnu9-ohpc
% ohpc_indent 0
% ohpc_command fi

If your system includes \InfiniBand{} and you enabled underlying support in
\S\ref{sec:add_ofed} and \S\ref{sec:addl_customizations}, an additional
MVAPICH2 family is available for use:

% begin_ohpc_run
% ohpc_validation_newline
% ohpc_command if [[ ${enable_ib} -eq 1 ]];then
% ohpc_indent 5
\begin{lstlisting}[language=bash]
[sms](*\#*) (*\install*) mvapich2-gnu9-ohpc
\end{lstlisting}
% ohpc_indent 0
% ohpc_command fi
% end_ohpc_run

Alternatively, if your system includes \IntelR{} \OmniPath{}, use the (\texttt{psm2})
variant of MVAPICH2 instead:

% begin_ohpc_run
% ohpc_command if [[ ${enable_opa} -eq 1 ]];then
% ohpc_indent 5
\begin{lstlisting}[language=bash]
[sms](*\#*) (*\install*) mvapich2-psm2-gnu9-ohpc
\end{lstlisting}
% ohpc_indent 0
% ohpc_command fi
% end_ohpc_run

An additional OpenMPI build variant is listed in Table~\ref{table:mpi} which
enables \href{https://pmix.github.io/pmix/}{\color{blue}{PMIx}} job launch
support for use with \SLURM{}. This optional variant is
available as \texttt{openmpi4-pmix-slurm-gnu9-ohpc}.


\subsection{Performance Tools} \label{sec:install_perf_tools}
\OHPC{} provides a variety of open-source tools to aid in application 
performance analysis (refer to Appendix~\ref{appendix:manifest} for a listing
of available packages). This group of tools can be installed as follows:

% begin_ohpc_run
% ohpc_comment_header Install Performance Tools \ref{sec:install_perf_tools}
\begin{lstlisting}[language=bash,keywords={},literate={-}{-}1]
# Install perf-tools meta-package
[sms](*\#*) (*\install*) ohpc-gnu9-perf-tools
\end{lstlisting}
% end_ohpc_run


\subsection{Setup default development environment}
System users often find it convenient to have a default development environment
in place so that compilation can be performed directly for parallel programs
requiring MPI. This setup can be conveniently enabled via modules and the \OHPC{}
modules environment is pre-configured to load an \texttt{ohpc} module on login
(if present). The following package install provides a default
environment that enables autotools, the \GNU{} compiler toolchain, and the
OpenMPI stack.

% begin_ohpc_run
\begin{lstlisting}[language=bash]
[sms](*\#*) (*\install*) lmod-defaults-gnu9-openmpi4-ohpc
\end{lstlisting}
% end_ohpc_run

\begin{center}
\begin{tcolorbox}[]
\small
\iftoggleverb{isx86}
If you want to change the default environment from the suggestion above, \OHPC{}
also provides the \GNU{} compiler toolchain with the MPICH and MVAPICH2 stacks:
\fi

\iftoggleverb{isaarch}
If you want to change the default environment from the suggestion above, \OHPC{}
also provides the \GNU{} compiler toolchain with the MPICH stack:
\fi

\begin{itemize*}
\item lmod-defaults-gnu9-mpich-ohpc
\iftoggleverb{isx86}
\item lmod-defaults-gnu9-mvapich2-ohpc
\fi
\end{itemize*}
\end{tcolorbox}
\end{center}


%\vspace*{0.2cm}
\subsection{3rd Party Libraries and Tools} \label{sec:3rdparty}
\input{common/third_party_libs_intro}

\begin{center}
\begin{tcolorbox}[]
\small
\OHPC{}-provided 3rd party builds are configured to be installed
into a common top-level repository so that they can be easily exported to
desired hosts within the cluster. This common top-level path
(\path{/opt/ohpc/pub}) was previously configured to be mounted on {\em
 compute} nodes in \S\ref{sec:master_customization}, so the packages will be
immediately available for use on the cluster after installation on the {\em
 master} host.
\end{tcolorbox}
\end{center}

%\iftoggle{isCentOS}{\clearpage}
%\nottoggle{isCentOS}{\clearpage}

For convenience, \OHPC{} provides package aliases for these 3rd party libraries
and utilities that can be used to install available libraries for use with the
GNU compiler family toolchain. For parallel libraries, aliases are grouped by
MPI family toolchain so that administrators can choose a subset should they
favor a particular MPI stack.  Please refer to Appendix~\ref{appendix:manifest}
for a more detailed listing of all available packages in each of these functional
areas. To install all available package offerings within \OHPC{}, issue the
following:

% begin_ohpc_run
% ohpc_comment_header Install 3rd Party Libraries and Tools \ref{sec:3rdparty}
\begin{lstlisting}[language=bash,keywords={},upquote=true,keepspaces]
# Install 3rd party libraries/tools meta-packages built with GNU toolchain
[sms](*\#*) (*\install*) ohpc-gnu9-serial-libs
[sms](*\#*) (*\install*) ohpc-gnu9-io-libs
[sms](*\#*) (*\install*) ohpc-gnu9-python-libs
[sms](*\#*) (*\install*) ohpc-gnu9-runtimes
\end{lstlisting}
% end_ohpc_run





% begin_ohpc_run
% ohpc_command if [[ ${enable_mpi_defaults} -eq 1 ]];then
% ohpc_indent 5
\begin{lstlisting}[language=bash,keywords={},upquote=true,keepspaces]
# Install parallel lib meta-packages for all available MPI toolchains
[sms](*\#*) (*\install*) ohpc-gnu9-mpich-parallel-libs
[sms](*\#*) (*\install*) ohpc-gnu9-openmpi4-parallel-libs
\end{lstlisting}
% ohpc_indent 0
% ohpc_command fi
% ohpc_command if [[ ${enable_ib} -eq 1 ]];then
% ohpc_indent 5
% ohpc_command (*\install*) ohpc-gnu9-mvapich2-parallel-libs
% ohpc_indent 0
% ohpc_command fi
% ohpc_command if [[ ${enable_opa} -eq 1 ]];then
% ohpc_indent 5
% ohpc_command (*\install*) ohpc-gnu9-mvapich2-parallel-libs
% ohpc_indent 0
% ohpc_command fi
% end_ohpc_run


\subsection{Optional Development Tool Builds} \label{sec:3rdparty_intel}
In addition to the 3rd party development libraries built using the open source
toolchains mentioned in \S\ref{sec:3rdparty}, \OHPC{} also provides {\em
  optional} compatible builds for use with the compilers and MPI stack included
in newer versions of the \IntelR{}~Parallel Studio XE software suite.  These
packages provide a similar hierarchical user
environment experience as other compiler and MPI families present in \OHPC{}.

To take advantage of the available builds, the Parallel Studio software suite
must be obtained and installed separately. Once installed locally, the \OHPC{}
compatible packages can be installed using standard package manager semantics.
Note that licenses are provided free of charge for many categories of use. In
particular, licenses for compilers and developments tools are provided at no
cost to academic researchers or developers contributing to open-source software
projects. More information on this program can be found at:

\begin{center}
  \href{https://software.intel.com/en-us/qualify-for-free-software}
       {\color{blue}{https://software.intel.com/en-us/qualify-for-free-software}}
\end{center}

\begin{center}
\begin{tcolorbox}[]
As noted in \S\ref{sec:master_customization}, the default installation path for
\OHPC{} (\texttt{/opt/ohpc/pub}) is exported over NFS from the {\em master} to the 
compute nodes, but the Parallel Studio installer defaults to a path of 
\texttt{/opt/intel}. To make the \IntelR{} compilers available to the compute 
nodes one must either customize the Parallel Studio installation path to be 
within \texttt{/opt/ohpc/pub}, or alternatively, add an additional NFS export
for \texttt{/opt/intel} that is mounted on desired compute nodes.
\end{tcolorbox}
\end{center}

\noindent To enable all 3rd party builds available in \OHPC{} that are compatible with
\IntelR{}~Parallel Studio, issue the following:

% begin_ohpc_run
% ohpc_comment_header Install Optional Development Tools for use with Intel Parallel Studio \ref{sec:3rdparty_intel}
% ohpc_command if [[ ${enable_intel_packages} -eq 1 ]];then
% ohpc_indent 5
\begin{lstlisting}[language=bash,keywords={},upquote=true,keepspaces]
# Install OpenHPC compatibility packages (requires prior installation of Parallel Studio)
[sms](*\#*) (*\install*) intel-compilers-devel-ohpc
[sms](*\#*) (*\install*) intel-mpi-devel-ohpc
\end{lstlisting}

% ohpc_command if [[ ${enable_opa} -eq 1 ]];then
% ohpc_indent 10
\begin{lstlisting}[language=bash,keywords={},upquote=true,keepspaces]
# Optionally, choose the Omni-Path enabled build for MVAPICH2. Otherwise, skip to retain IB variant
[sms](*\#*) (*\install*) mvapich2-psm2-intel-ohpc
\end{lstlisting}
% ohpc_indent 5
% ohpc_command fi

\iftoggle{isSLES_ww_slurm_x86}{\clearpage}

\begin{lstlisting}[language=bash,keywords={},upquote=true,keepspaces]
# Install 3rd party libraries/tools meta-packages built with Intel toolchain
[sms](*\#*) (*\install*) ohpc-intel-serial-libs
[sms](*\#*) (*\install*) ohpc-intel-io-libs
[sms](*\#*) (*\install*) ohpc-intel-perf-tools
[sms](*\#*) (*\install*) ohpc-intel-python-libs
[sms](*\#*) (*\install*) ohpc-intel-runtimes
[sms](*\#*) (*\install*) ohpc-intel-mpich-parallel-libs
[sms](*\#*) (*\install*) ohpc-intel-mvapich2-parallel-libs
[sms](*\#*) (*\install*) ohpc-intel-openmpi3-parallel-libs
[sms](*\#*) (*\install*) ohpc-intel-impi-parallel-libs
\end{lstlisting}
% ohpc_indent 0
% ohpc_command fi
% end_ohpc_run



\vspace*{0.2cm}
\section{Resource Manager Startup} \label{sec:rms_startup}
\input{common/slurm_startup_stateful}

\section{Run a Test Job} \label{sec:test_job}
With the resource manager enabled for production usage, users should now be
able to run jobs. To demonstrate this, we will add a ``test'' user on the {\em master}
host that can be used to run an example job.

% begin_ohpc_run
\begin{lstlisting}[language=bash,keywords={}]
[sms](*\#*) useradd -m test
\end{lstlisting}
% end_ohpc_run

Next, the user's credentials need to be distributed across the cluster.
\xCAT{}'s \texttt{xdcp} has a merge functionality that adds new entries into
credential files on {\em compute} nodes: 

% begin_ohpc_run
\begin{lstlisting}[language=bash,keywords={}]
# Create a sync file for pushing user credentials to the nodes
[sms](*\#*) echo "MERGE:" > syncusers
[sms](*\#*) echo "/etc/passwd -> /etc/passwd" >> syncusers
[sms](*\#*) echo "/etc/group -> /etc/group"       >> syncusers
[sms](*\#*) echo "/etc/shadow -> /etc/shadow" >> syncusers
# Use xCAT to distribute credentials to nodes
[sms](*\#*) xdcp compute -F syncusers
\end{lstlisting}
% end_ohpc_run

\nottoggle{isxCATstateful}{Alternatively, the \texttt{updatenode compute -f} command
can be used. This re-synchronizes (i.e. copies) all the files defined in the
\texttt{syncfile} setup in Section \ref{sec:file_import}. \\ }  

~\\
\input{common/prun}

\iftoggle{isSLES_ww_slurm_x86}{\clearpage}
%\iftoggle{isxCAT}{\clearpage}

\subsection{Interactive execution}
To use the newly created ``test'' account to compile and execute the
application {\em interactively} through the resource manager, execute the
following (note the use of \texttt{prun} for parallel job launch which summarizes
the underlying native job launch mechanism being used):

\begin{lstlisting}[language=bash,keywords={}]
# Switch to "test" user
[sms](*\#*) su - test

# Compile MPI "hello world" example
[test@sms ~]$ mpicc -O3 /opt/ohpc/pub/examples/mpi/hello.c

# Submit interactive job request and use prun to launch executable
[test@sms ~]$ srun -n 8 -N 2 --pty /bin/bash

[test@c1 ~]$ prun ./a.out

[prun] Master compute host = c1
[prun] Resource manager = slurm
[prun] Launch cmd = mpiexec.hydra -bootstrap slurm ./a.out

 Hello, world (8 procs total)
    --> Process #   0 of   8 is alive. -> c1
    --> Process #   4 of   8 is alive. -> c2
    --> Process #   1 of   8 is alive. -> c1
    --> Process #   5 of   8 is alive. -> c2
    --> Process #   2 of   8 is alive. -> c1
    --> Process #   6 of   8 is alive. -> c2
    --> Process #   3 of   8 is alive. -> c1
    --> Process #   7 of   8 is alive. -> c2
\end{lstlisting}

\begin{center}
\begin{tcolorbox}[]
The following table provides approximate command equivalences between SLURM and
PBS Pro:

\vspace*{0.15cm}
\input common/rms_equivalence_table
\end{tcolorbox}
\end{center}
\nottoggle{isCentOS}{\clearpage}

\iftoggle{isCentOS}{\clearpage}

\subsection{Batch execution}

For batch execution, \OHPC{} provides a simple job script for reference (also
housed in the \path{/opt/ohpc/pub/examples} directory. This example script can
be used as a starting point for submitting batch jobs to the resource manager
and the example below illustrates use of the script to submit a batch job for
execution using the same executable referenced in the previous interactive example.

\begin{lstlisting}[language=bash,keywords={}]
# Copy example job script
[test@sms ~]$ cp /opt/ohpc/pub/examples/slurm/job.mpi .

# Examine contents (and edit to set desired job sizing characteristics)
[test@sms ~]$ cat job.mpi
#!/bin/bash

#SBATCH -J test               # Job name
#SBATCH -o job.%j.out         # Name of stdout output file (%j expands to %jobId)
#SBATCH -N 2                  # Total number of nodes requested
#SBATCH -n 16                 # Total number of mpi tasks #requested
#SBATCH -t 01:30:00           # Run time (hh:mm:ss) - 1.5 hours

# Launch MPI-based executable

prun ./a.out

# Submit job for batch execution
[test@sms ~]$ sbatch job.mpi
Submitted batch job 339
\end{lstlisting}

\begin{center}
\begin{tcolorbox}[]
\small
The use of the \texttt{\%j} option in the example batch job script shown is a convenient
way to track application output on an individual job basis. The \texttt{\%j} token
is replaced with the \SLURM{} job allocation number once assigned (job~\#339 in
this example).
\end{tcolorbox}
\end{center}




\clearpage
\appendix
{\bf \LARGE \centerline{Appendices}} \vspace*{0.2cm}

\addcontentsline{toc}{section}{Appendices}
\renewcommand{\thesubsection}{\Alph{subsection}}

\input{common/automation_appendix}
\input{common/upgrade_stateful}
\input{common/test_suite}
\input{common/customization_appendix_centos}
../slurm/manifest.tex
\input{common/signature}


\end{document}

