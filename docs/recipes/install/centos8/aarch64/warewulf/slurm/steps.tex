\documentclass[letterpaper]{article}
\usepackage{common/ohpc-doc}
\setcounter{secnumdepth}{5}
\setcounter{tocdepth}{5}

% Include git variables
\input{vc.tex}

% Define Base OS and other local macros
\newcommand{\baseOS}{CentOS8.1}
\newcommand{\OSRepo}{CentOS\_8.1}
\newcommand{\OSTree}{CentOS\_8}
\newcommand{\OSTag}{el8}
\newcommand{\baseos}{centos8.1}
\newcommand{\baseosshort}{centos8}
\newcommand{\provisioner}{Warewulf}
\newcommand{\provheader}{\provisioner{}}
\newcommand{\rms}{SLURM}
\newcommand{\rmsshort}{slurm}
\newcommand{\arch}{aarch64}

% Define package manager commands
\newcommand{\pkgmgr}{yum}
\newcommand{\addrepo}{wget -P /etc/yum.repos.d}
\newcommand{\chrootaddrepo}{wget -P \$CHROOT/etc/yum.repos.d}
\newcommand{\clean}{yum clean expire-cache}
\newcommand{\chrootclean}{yum --installroot=\$CHROOT clean expire-cache}
\newcommand{\install}{yum -y install}
\newcommand{\chrootinstall}{yum -y --installroot=\$CHROOT install}
\newcommand{\groupinstall}{yum -y groupinstall}
\newcommand{\groupchrootinstall}{yum -y --installroot=\$CHROOT groupinstall}
\newcommand{\remove}{yum -y remove}
\newcommand{\upgrade}{yum -y upgrade}
\newcommand{\chrootupgrade}{yum -y --installroot=\$CHROOT upgrade}
\newcommand{\tftppkg}{syslinux-tftpboot}

% boolean for os-specific formatting
\toggletrue{isaarch}
\toggletrue{isCentOS}
\toggletrue{isCentOS_ww_slurm_aarch}
\toggletrue{isSLURM}
\toggletrue{isWarewulf}

\begin{document}
\graphicspath{{common/figures/}}
\thispagestyle{empty}

% Title Page --------------------------------------------------------
\input{common/title}
% Disclaimer
\newpage

\vspace*{3.0cm}
\noindent {\Large \color{logoblue} \fontfamily{phv}\selectfont Legal Notice} \\ 

\vspace*{0.5cm}

\noindent Copyright {\small\copyright} 2016-2020, OpenHPC, a Linux Foundation
Collaborative Project. All rights reserved. \\

\vspace*{0.1cm}

\noindent \begin{tabular}{cp{10cm}}
\raisebox{-.75\height}{\includegraphics[width=0.22\textwidth]{cc_by}} &
This documentation is licensed under the Creative Commons Attribution 4.0 International
License. To view a copy of this license, visit
\href{http://creativecommons.org/licenses/by/4.0}{\color{blue}{http://creativecommons.org/licenses/by/4.0}}. \\
\end{tabular}


\vspace*{1.5cm}

{\footnotesize

\noindent Intel, the Intel logo, and other Intel marks are trademarks of Intel
Corporation in the U.S. and/or other countries. \\
\iftoggleverb{ispbs}
\noindent Altair, the Altair logo, PBS Professional, and other Altair marks are
trademarks of Altair Engineering, Inc. in the U.S. and/or other countries. \\
\fi
\noindent *Other names and brands may be claimed as the property of others. \\



}
 

\newpage
\tableofcontents
\newpage

% Introduction  ----------------------------------------------------


\section{Introduction} \label{sec:introduction}
\input{common/install_header}
\input{common/intro} \\

\input{common/base_edition/edition}
\subsection{Target Audience}

This guide is targeted at experienced \Linux{} system administrators for HPC
environments. Knowledge of software package management, system networking, and
PXE booting is assumed. Command-line input examples are highlighted throughout
this guide via the following syntax:

\begin{lstlisting}[language=bash,literate={-}{-}1,keywords={},upquote=true]
[sms](*\#*) echo "OpenHPC hello world"
\end{lstlisting}

Unless specified otherwise, the examples presented are executed with
elevated (root) privileges. The examples also presume use of the BASH login
shell, though the equivalent commands in other shells can be substituted.
In addition to specific command-line instructions called out in this guide, an
alternate convention is used to highlight potentially useful tips or optional
configuration options. These tips are highlighted via the following format:

\begin{center}
\begin{tcolorbox}[]
\small  Training is everything. The peach was once a bitter almond; cauliflower is nothing but cabbage with a college education. --Mark Twain
\end{tcolorbox}
\end{center}


\input{common/requirements}
\input{common/inputs}

% begin_ohpc_run
% ohpc_validation_newline
% ohpc_validation_comment Verify OpenHPC repository has been enabled before proceeding
% ohpc_validation_newline
% ohpc_command yum repolist | grep -q OpenHPC
% ohpc_command if [ $? -ne 0 ];then
% ohpc_command    echo "Error: OpenHPC repository must be enabled locally"
% ohpc_command    exit 1
% ohpc_command fi
% end_ohpc_run

% Base Operating System --------------------------------------------

\section{Install Base Operating System (BOS)}
\input{common/bos}

% begin_ohpc_run
% ohpc_validation_newline
% ohpc_validation_comment Disable firewall 
\begin{lstlisting}[language=bash,keywords={}]
[sms](*\#*) systemctl disable firewalld
[sms](*\#*) systemctl stop firewalld
\end{lstlisting}
% end_ohpc_run

% ------------------------------------------------------------------

\section{Install \OHPC{} Components} \label{sec:basic_install}
\input{common/install_ohpc_components_intro.tex}

\subsection{Enable \OHPC{} repository for local use} \label{sec:enable_repo}
\input{common/enable_ohpc_repo}

In addition to the \OHPC{} package repository, the {\em master} host also
requires access to the standard base OS distro repositories in order to resolve
necessary dependencies. For \baseOS{}, the requirements are to have access to
both the base OS and EPEL repositories for which mirrors are freely available online:

\begin{itemize*}
\item CentOS-7 - Base 7.7.1908
  (e.g. \href{http://mirror.centos.org/altarch/7.7.1908/os/aarch64/}
  {\color{blue}{http://mirror.centos.org/altarch/7.7.1908/os/aarch64/}} )
  \item EPEL 7
    (e.g. \href{http://download.fedoraproject.org/pub/epel/7/aarch64}
                            {\color{blue}{http://download.fedoraproject.org/pub/epel/7/aarch64}})
\end{itemize*}

\input{common/automation}

\subsection{Add provisioning services on {\em master} node} \label{sec:add_provisioning}
\input{common/install_provisioning_intro}
\input{common/enable_pxe}
HPC systems rely on synchronized clocks throughout the system and the
NTP protocol can be used to facilitate this synchronization. To enable NTP
services on the SMS host with a specific server \texttt{\$\{ntp\_server\}},
issue the following:

% begin_ohpc_run
% ohpc_validation_comment Enable NTP services on SMS host
\begin{lstlisting}[language=bash,literate={-}{-}1,keywords={},upquote=true,keepspaces]
[sms](*\#*) systemctl enable chronyd.service
[sms](*\#*) echo "server ${ntp_server}" >> /etc/ntp.conf
[sms](*\#*) systemctl restart ntpd
\end{lstlisting}
% end_ohpc_run


\subsection{Add resource management services on {\em master} node} \label{sec:add_rm}
\input{common/install_slurm}

%% Add if/when IB is available for testing
%% \subsection{Optionally add \InfiniBand{} support services on {\em master} node} \label{sec:add_ofed}
%% \input{common/ibsupport_sms_centos}

%\vspace*{-0.15cm}
\subsection{Complete basic Warewulf setup for {\em master} node} \label{sec:setup_ww}
\input{common/warewulf_setup}
% begin_ohpc_run
% ohpc_comment_header Complete basic Warewulf setup for master node \ref{sec:setup_ww}
%\begin{verbatim}

\begin{lstlisting}[language=bash,literate={-}{-}1,keywords={},upquote=true,keepspaces]
# Configure Warewulf provisioning to use desired internal interface
[sms](*\#*) perl -pi -e "s/device = eth1/device = ${sms_eth_internal}/" /etc/warewulf/provision.conf

# Enable tftp service for compute node image distribution
[sms](*\#*) perl -pi -e "s/^\s+disable\s+= yes/ disable = no/" /etc/xinetd.d/tftp

# Enable internal interface for provisioning
[sms](*\#*) ifconfig ${sms_eth_internal} ${sms_ip} netmask ${internal_netmask} up

# Restart/enable relevant services to support provisioning
[sms](*\#*) systemctl enable httpd.service
[sms](*\#*) systemctl restart httpd
[sms](*\#*) systemctl enable dhcpd.service
[sms](*\#*) systemctl enable tftp
[sms](*\#*) systemctl restart tftp
\end{lstlisting}
%\end{verbatim}
% end_ohpc_run


\subsection{Define {\em compute} image for provisioning}
With the provisioning services enabled, the next step is to define and
customize a system image that can subsequently be used to provision one or more
{\em compute} nodes. The following subsections highlight this process.

\subsubsection{Build initial BOS image} \label{sec:assemble_bos}
The \OHPC{} build of \Warewulf{} includes specific enhancements enabling support for
\baseOS{}. The following steps illustrate the process to build a minimal, default
image for use with \Warewulf{}. We begin by defining a directory structure on the 
{\em master} host that will represent the root filesystem of the compute node. The 
default location for this example is in
\texttt{/opt/ohpc/admin/images/\baseos{}}.

\begin{center}
  \begin{tcolorbox}[]
    \small \Warewulf{} is configured by default to access an external
    repository (mirror.centos.org) during the \texttt{wwmkchroot} process.  If
    the master host cannot reach the public CentOS mirrors, or if you prefer to
    access a locally cached mirror, set the \texttt{\$\{YUM\_MIRROR\}}
    environment variable to your desired repo location {\em prior} to running
    the \texttt{wwmkchroot} command below. For example:

% begin_ohpc_run
% ohpc_command if [ ! -z ${BOS_MIRROR+x} ]; then
% ohpc_indent 5
\begin{lstlisting}[language=bash,keywords={}]
# Override default OS repository (optional) - set YUM_MIRROR variable to desired repo location
[sms](*\#*) export YUM_MIRROR=${BOS_MIRROR}
\end{lstlisting}
% ohpc_indent 0
% ohpc_command fi
% end_ohpc_run

\end{tcolorbox}
\end{center}

% begin_ohpc_run
% ohpc_comment_header Create compute image for Warewulf \ref{sec:assemble_bos}
\begin{lstlisting}[language=bash,literate={-}{-}1,keywords={},upquote=true,keepspaces,literate={BOSVER}{\baseos{}}1]
# Define chroot location 
[sms](*\#*) export CHROOT=/opt/ohpc/admin/images/BOSVER

# Build initial chroot image
[sms](*\#*) wwmkchroot -v centos-8 $CHROOT
# Enable OpenHPC and EPEL repos inside chroot
[sms](*\#*) dnf -y --installroot $CHROOT install epel-release
[sms](*\#*) cp -p /etc/yum.repos.d/OpenHPC.repo $CHROOT/etc/yum.repos.d
\end{lstlisting}
% end_ohpc_run


\subsubsection{Add \OHPC{} components} \label{sec:add_components}
\input{common/add_to_compute_chroot_intro}

% begin_ohpc_run
% ohpc_validation_comment Add OpenHPC components to compute instance
\begin{lstlisting}[language=bash,literate={-}{-}1,keywords={},upquote=true]
# Add Slurm client support meta-package
[sms](*\#*) (*\chrootinstall*) ohpc-slurm-client

# Add Network Time Protocol (NTP) support
[sms](*\#*) (*\chrootinstall*) ntp

# Add kernel drivers
[sms](*\#*) (*\chrootinstall*) kernel

# Include modules user environment
[sms](*\#*) (*\chrootinstall*) lmod-ohpc
\end{lstlisting}
% end_ohpc_run

\subsubsection{Customize system configuration} \label{sec:master_customization}
Prior to assembling the image, it is advantageous to perform any additional
customization within the chroot environment created for the desired {\em
 compute} instance. The following steps document the process to add a local
{\em ssh} key created by \Warewulf{} to support remote access, identify the
resource manager server, configure NTP for compute resources, and enable \NFS{}
mounting of a \$HOME file system and the public \OHPC{} install path
(\texttt{/opt/ohpc/pub}) that will be hosted by the {\em master} host in this
example configuration.

\iftoggleverb{isCentOS_ww_pbs_x86}
\vspace*{0.15cm}
%\clearpage
\else
\vspace*{0.15cm}
\fi

% begin_ohpc_run
% ohpc_comment_header Customize system configuration \ref{sec:master_customization}
\begin{lstlisting}[language=bash,literate={-}{-}1,keywords={},upquote=true]
# Initialize warewulf database and ssh_keys
[sms](*\#*) wwinit database
[sms](*\#*) wwinit ssh_keys

# Add NFS client mounts of /home and /opt/ohpc/pub to base image
[sms](*\#*) echo "${sms_ip}:/home /home nfs nfsvers=3,nodev,nosuid 0 0" >> $CHROOT/etc/fstab
[sms](*\#*) echo "${sms_ip}:/opt/ohpc/pub /opt/ohpc/pub nfs nfsvers=3,nodev 0 0" >> $CHROOT/etc/fstab

# Export /home and OpenHPC public packages from master server
[sms](*\#*) echo "/home *(rw,no_subtree_check,fsid=10,no_root_squash)" >> /etc/exports
[sms](*\#*) echo "/opt/ohpc/pub *(ro,no_subtree_check,fsid=11)" >> /etc/exports
[sms](*\#*) exportfs -a
[sms](*\#*) systemctl restart nfs-server
[sms](*\#*) systemctl enable nfs-server

# Enable NTP time service on computes and identify master host as local NTP server
[sms](*\#*) chroot $CHROOT systemctl enable ntpd
[sms](*\#*) echo "server ${sms_ip}" >> $CHROOT/etc/ntp.conf

\end{lstlisting}
% end_ohpc_run



% Additional commands when additional computes are requested

% begin_ohpc_run
% ohpc_validation_newline
% ohpc_validation_comment Update basic slurm configuration if additional computes defined
% ohpc_command if [ ${num_computes} -gt 4 ];then
% ohpc_command    perl -pi -e "s/^NodeName=(\S+)/NodeName=${compute_prefix}[1-${num_computes}]/" /etc/slurm/slurm.conf
% ohpc_command    perl -pi -e "s/^PartitionName=normal Nodes=(\S+)/PartitionName=normal Nodes=${compute_prefix}[1-${num_computes}]/" /etc/slurm/slurm.conf

% ohpc_command    perl -pi -e "s/^NodeName=(\S+)/NodeName=${compute_prefix}[1-${num_computes}]/" $CHROOT/etc/slurm/slurm.conf
% ohpc_command    perl -pi -e "s/^PartitionName=normal Nodes=(\S+)/PartitionName=normal Nodes=${compute_prefix}[1-${num_computes}]/" $CHROOT/etc/slurm/slurm.conf
% ohpc_command fi
% end_ohpc_run

%\clearpage
\subsubsection{Additional Customization ({\em optional})} \label{sec:addl_customizations}
This section highlights common additional customizations that can {\em
optionally} be applied to the local cluster environment. These customizations
include:

\begin{multicols}{2}
\begin{itemize*}
\iftoggleverb{isx86}
\item Add InfiniBand or Omni-Path drivers
\item Increase memlock limits
\fi

\nottoggle{ispbs}{\item Restrict ssh access to compute resources}

\iftoggleverb{isx86}
\item Add \beegfs{} client
\item Add \Lustre{} client
\fi

\iftoggle{isWarewulf}{\item Enable syslog forwarding}

\item Add \Nagios{} Core monitoring
\item Add \Ganglia{} monitoring
\item Add \Sensys{} monitoring
\item Add \clustershell{}
\item Add \mrsh{}
\item Add \genders{}
%%\item Add \powerman{}
\item Add \conman{}  
\item Add \GEOPM{}
\end{itemize*}
\end{multicols}

\noindent Details on the steps required for each of these customizations are
discussed further in the following sections.


%% Add if/when IB is available for testing
%% \paragraph{Increase locked memory limits}
%% \input{common/memlimits}

\paragraph{Enable ssh control via resource manager} 
\input{common/slurm_pam}

%%\paragraph{Add \Lustre{} client} \label{sec:lustre_client}
%%\input{common/lustre-client}
%%\input{common/lustre-client-centos}
%%\vspace*{0.5cm}
%%\input{common/lustre-client-post}

%\clearpage
\paragraph{Enable forwarding of system logs} \label{sec:add_syslog}
\input{common/syslog}

\paragraph{Add \Nagios{} monitoring}
\input{common/nagios}

\clearpage
\paragraph{Add \Ganglia{} monitoring}
\input{common/ganglia}

\paragraph{Add \clustershell{}}
\clustershell{} is an event-based Python library to execute commands in parallel
across cluster nodes. Installation and basic configuration defining three node
groups ({\em adm}, {\em compute}, and {\em all}) is as follows:

% begin_ohpc_run
% ohpc_validation_newline
% ohpc_command if [[ ${enable_clustershell} -eq 1 ]];then
% ohpc_indent 5
% ohpc_validation_comment Install clustershell
\begin{lstlisting}[language=bash,keywords={},upquote=true]
# Install ClusterShell
[sms](*\#*) (*\install*) clustershell

# Setup node definitions
[sms](*\#*) cd /etc/clustershell/groups.d
[sms](*\#*) mv local.cfg local.cfg.orig
[sms](*\#*) echo "adm: ${sms_name}" > local.cfg
[sms](*\#*) echo "compute: ${compute_prefix}[1-${num_computes}]" >> local.cfg
[sms](*\#*) echo "all: @adm,@compute" >> local.cfg
\end{lstlisting}
% ohpc_indent 0
% ohpc_command fi
% end_ohpc_run



\clearpage
\paragraph{Add \mrsh{}}
\input{common/mrsh}

\paragraph{Add \genders{}}
\input{common/genders}

\paragraph{Add \conman{}} \label{sec:add_conman}
\input{common/conman}

\paragraph{Add \nhc{}} \label{sec:add_nhc}
Resource managers often provide for a periodic "node health check" to be
performed on each compute node to verify that the node is working
properly. Nodes which are determined to be "unhealthy" can be marked as down or
offline so as to prevent jobs from being scheduled or run on them. This helps
increase the reliability and throughput of a cluster by reducing preventable
job failures due to misconfiguration, hardware failure, etc. OpenHPC
distributes \nhc{} to fulfill this requirement.

In a typical scenario, the \nhc{} driver script is run periodically on each compute
node by the resource manager client daemon. It loads its
configuration file to determine which checks are to be run on the current node
(based on its hostname). Each matching check is run, and if a failure is
encountered, \nhc{} will exit with an error message describing the problem. It can
also be configured to mark nodes offline so that the scheduler will not assign
jobs to bad nodes, reducing the risk of system-induced job failures. 

% begin_ohpc_run
% ohpc_validation_newline
% ohpc_validation_comment Optionally, enable nhc and configure
\begin{lstlisting}[language=bash,keywords={},upquote=true]
# Install NHC on master and compute nodes
[sms](*\#*) (*\install*) nhc-ohpc
[sms](*\#*) (*\chrootinstall*) nhc-ohpc
\end{lstlisting}
% end_ohpc_run


% begin_ohpc_run
% ohpc_validation_newline
\begin{lstlisting}[language=bash,keywords={},upquote=true]
# Register as SLURM's health check program
[sms](*\#*) echo "HealthCheckProgram=/usr/sbin/nhc" >> /etc/slurm/slurm.conf
[sms](*\#*) echo "HealthCheckInterval=300" >> /etc/slurm/slurm.conf  # execute every five minutes
\end{lstlisting}
% end_ohpc_run




\subsubsection{Import files} \label{sec:file_import}
\input{common/import_ww_files}
\input{common/import_ww_files_slurm}
%% \input{common/import_ww_files_ib_centos}
\subsection{Finalizing provisioning configuration} \label{sec:assemble_bootstrap}

\Warewulf{} employs a two-stage boot process for provisioning nodes via
creation of a bootstrap image that is used to initialize the process, and a virtual node
file system capsule containing the full system image. This section highlights
creation of the necessary provisioning images, followed by the registration of
desired compute nodes.

\subsubsection{Assemble bootstrap image}

The bootstrap image includes the runtime kernel and associated modules, as well
as some simple scripts to complete the provisioning process. The
following commands highlight the inclusion of additional drivers and creation
of the bootstrap image based on the running kernel.

%\iftoggle{isCentOS_ww_slurm_aarch}{\clearpage}

% begin_ohpc_run
% ohpc_comment_header Assemble bootstrap image \ref{sec:assemble_bootstrap}
\begin{lstlisting}[language=bash,literate={-}{-}1,keywords={},upquote=true]
# (Optional) Include drivers from kernel updates;  needed if enabling additional kernel modules on computes
[sms](*\#*) export WW_CONF=/etc/warewulf/bootstrap.conf
[sms](*\#*) echo "drivers += updates/kernel/" >> $WW_CONF

# (Optional) Include overlayfs drivers; needed by Singularity
[sms](*\#*) echo "drivers += overlay" >> $WW_CONF

# Build bootstrap image
[sms](*\#*) wwbootstrap `uname -r`
\end{lstlisting}
% end_ohpc_run

\subsubsection{Assemble Virtual Node File System (VNFS) image}

With the local site customizations in place, the following step uses the
\texttt{wwvnfs} command to assemble a VNFS capsule from the chroot environment
defined for the {\em compute} instance. 

% begin_ohpc_run
% ohpc_validation_comment Assemble VNFS
\begin{lstlisting}[language=bash,literate={-}{-}1,keywords={},upquote=true]
[sms](*\#*) wwvnfs --chroot $CHROOT
\end{lstlisting}
% end_ohpc_run

\iftoggle{isCentOS_ww_slurm_aarch}{\vspace*{0.4cm}}

\subsubsection{Register nodes for provisioning}

In preparation for provisioning, we can now define the desired network settings
for four example compute nodes with the underlying provisioning system and
restart the \texttt{dhcp} service. Note the use of variable names for the
desired compute hostnames, node IPs, and MAC addresses which should be modified
to accommodate local settings and hardware.  By default, \Warewulf{} uses
network interface names of the \texttt{eth\#} variety and adds kernel boot
arguments to maintain this scheme on newer kernels. Consequently, when specifying
the desired provisioning interface via the \texttt{\$eth\_provision} variable,
it should follow this convention. Alternatively, if you prefer to use the
predictable network interface naming scheme (e.g. names like \texttt{en4s0f0}),
additional steps are included to alter the default kernel boot arguments and take
the \texttt{eth\#} named interface down after bootstrapping so the normal init
process can bring it up again using the desired name.

\iftoggleverb{isx86}
Also included in these steps are commands
to enable \Warewulf{} to manage IPoIB settings and corresponding definitions of
IPoIB addresses for the compute nodes. This is typically optional unless you
are planning to include a \Lustre{} client mount over \InfiniBand{}.
\fi
The final step
in this process associates the VNFS image assembled in previous steps with the
newly defined compute nodes, utilizing the user credential files and munge key
that were imported in \S\ref{sec:file_import}.



\input{common/add_ww_hosts_intro}
% begin_ohpc_run
% ohpc_validation_comment Add hosts to cluster (Cont.)
\begin{lstlisting}[language=bash,keywords={},upquote=true,basicstyle=\footnotesize\ttfamily,literate={BOSVER}{\baseos{}}1]
# Define provisioning image for hosts
[sms](*\#*) wwsh -y provision set "${compute_regex}" --vnfs=BOSVER --bootstrap=`uname -r` \
    --files=dynamic_hosts,passwd,group,shadow,slurm.conf,munge.key,network 
\end{lstlisting}



\input{common/add_ww_hosts_finalize}

\vspace*{0.2cm}
\subsubsection{Optional kernel arguments} \label{sec:optional_kargs}
The Charliecloud container runtime requires enabling the user namespaces mapping
option. This allows applications to run with root privilege inside a container, 
but have them run as a different, typically non-privileged, user on the host.
Though generally regarded as mature and safe, CentOS considers this a Technology
Preview, so it must be manually enabled with kernel arguments. We demonstrate
enabling on compute nodes, but it would also be required on the SMS if you wish
to build and run containers there.

% begin_ohpc_run
% ohpc_validation_newline
% ohpc_validation_comment Optionally, enable user namespaces
\begin{lstlisting}[language=bash,keywords={},upquote=true]
# Define node kernel arguments to support user namespaces
[sms](*\#*) export kargs="${kargs} namespace.unpriv_enable=1"

# Increase per-user limit on the number of user namespaces that may be created
[sms](*\#*) echo "user.max_user_namespaces=15076" >> $CHROOT/etc/sysctl.conf
# rebuild VNFS
[sms](*\#*) wwvnfs --chroot $CHROOT
\end{lstlisting}
% end_ohpc_run

\begin{center}
\begin{tcolorbox}[]
\small
Typical Charliecloud workflows are based around Docker containers, but it is not
strictly necessary to install Docker itself on the HPC resource. A common
pattern is to build the Docker container on a laptop or VM and upload the result
to the cluster for use with Charliecloud. More information can be found at
\href{https://hpc.github.io/charliecloud/}
    {\color{blue}{https://hpc.github.io/charliecloud/}}
\end{tcolorbox}
\end{center}

If you chose to enable \conman{} in \S\ref{sec:add_conman}, additional
warewulf configuration is needed as follows:
% begin_ohpc_run
% ohpc_validation_newline
% ohpc_validation_comment Optionally, enable console redirection
% ohpc_command if [[ ${enable_ipmisol} -eq 1 ]];then
% ohpc_indent 5
\begin{lstlisting}[language=bash,keywords={},upquote=true]
# Define node kernel arguments to support SOL console
[sms](*\#*) wwsh -y provision set "${compute_regex}" --console=ttyS1,115200
\end{lstlisting}
% ohpc_indent 0
% ohpc_command fi
% end_ohpc_run

%%% If any components have added to the boot time kernel command line argumenst for the compute nodes,
%%% the following command is required to store the configuration in Warewulf:
%%% % ohpc_validation_newline
%%% % ohpc_validation_comment Optionally, add arguments to bootstrap kernel
%%% % ohpc_command if [[ ${enable_kargs} -eq 1 ]]; then
%%% \begin{lstlisting}[language=bash,keywords={},upquote=true,basicstyle=\footnotesize\ttfamily]
%%% # Set optional compute node kernel command line arguments.
%%% [sms](*\#*) wwsh -y provision set "${compute_regex}" --kargs="${kargs}"
%%% \end{lstlisting}
%%% % ohpc_command fi

\noindent If any components have added to the boot time kernel command line arguments for the compute nodes,
the following command is required to store the configuration in Warewulf:
% begin_ohpc_run
% ohpc_validation_newline
% ohpc_validation_comment Optionally, add arguments to bootstrap kernel
% ohpc_command if [[ ${enable_kargs} -eq 1 ]]; then
\begin{lstlisting}[language=bash,keywords={},upquote=true,basicstyle=\footnotesize\ttfamily]
# Set optional compute node kernel command line arguments.
[sms](*\#*) wwsh -y provision set "${compute_regex}" --kargs="${kargs}"
\end{lstlisting}
% ohpc_command fi
% end_ohpc_run


\subsubsection{Optionally configure stateful provisioning}
\input{common/stateful}

\vspace*{-0.1cm}
\subsection{Boot compute nodes} \label{sec:boot_computes}
\input{common/reset_computes} 

\vspace*{-0.50cm}
\section{Install \OHPC{} Development Components} \label{sec:install_dev}
\input{common/dev_intro.tex}

\vspace*{-0.15cm}
\subsection{Development Tools} \label{sec:install_dev_tools}
\input{common/dev_tools}

\vspace*{-0.15cm}
\subsection{Compilers} \label{sec:install_compilers}
\OHPC{} presently packages the \GNU{} compiler toolchain integrated with the 
underlying modules-environment system in a hierarchical fashion. The modules
system will conditionally present compiler-dependent software based on the
toolchain currently loaded. 

% begin_ohpc_run
% ohpc_comment_header Install Compilers \ref{sec:install_compilers}
\begin{lstlisting}[language=bash]
[sms](*\#*) (*\install*) gnu9-compilers-ohpc
\end{lstlisting}
% end_ohpc_run

The llvm compiler toolchains are also provided as a standalone additional
compiler family (ie. users can easily switch between gcc/clang environments),
but we do not provide the full complement of downstream library builds.

% begin_ohpc_run
% ohpc_comment_header Install llvm Compilers
\begin{lstlisting}[language=bash]
[sms](*\#*) (*\install*) llvm5-compilers-ohpc
\end{lstlisting}
% end_ohpc_run


%\clearpage
\subsection{MPI Stacks} \label{sec:mpi}
For MPI development and runtime support, \OHPC{} provides pre-packaged builds
for two MPI families that are compatible with ethernet fabrics. These MPI
stacks can be installed as follows:

% begin_ohpc_run
% ohpc_comment_header Install MPI Stacks \ref{sec:mpi}
% ohpc_command if [[ ${enable_mpi_defaults} -eq 1 ]];then
% ohpc_indent 5
\begin{lstlisting}[language=bash]
[sms](*\#*) (*\install*) openmpi4-gnu9-ohpc mpich-gnu9-ohpc
\end{lstlisting}
% ohpc_indent 0
% ohpc_command fi
% end_ohpc_run






\subsection{Performance Tools} \label{sec:install_perf_tools}
\OHPC{} provides a variety of open-source tools to aid in application 
performance analysis (refer to Appendix~\ref{appendix:manifest} for a listing
of available packages). This group of tools can be installed as follows:

% begin_ohpc_run
% ohpc_comment_header Install Performance Tools \ref{sec:install_perf_tools}
\begin{lstlisting}[language=bash,keywords={},literate={-}{-}1]
# Install perf-tools meta-package
[sms](*\#*) (*\install*) ohpc-gnu9-perf-tools
\end{lstlisting}
% end_ohpc_run


\subsection{Setup default development environment}
System users often find it convenient to have a default development environment
in place so that compilation can be performed directly for parallel programs
requiring MPI. This setup can be conveniently enabled via modules and the \OHPC{}
modules environment is pre-configured to load an \texttt{ohpc} module on login
(if present). The following package install provides a default
environment that enables autotools, the \GNU{} compiler toolchain, and the
OpenMPI stack.

% begin_ohpc_run
\begin{lstlisting}[language=bash]
[sms](*\#*) (*\install*) lmod-defaults-gnu9-openmpi4-ohpc
\end{lstlisting}
% end_ohpc_run

\begin{center}
\begin{tcolorbox}[]
\small
\iftoggleverb{isx86}
If you want to change the default environment from the suggestion above, \OHPC{}
also provides the \GNU{} compiler toolchain with the MPICH and MVAPICH2 stacks:
\fi

\iftoggleverb{isaarch}
If you want to change the default environment from the suggestion above, \OHPC{}
also provides the \GNU{} compiler toolchain with the MPICH stack:
\fi

\begin{itemize*}
\item lmod-defaults-gnu9-mpich-ohpc
\iftoggleverb{isx86}
\item lmod-defaults-gnu9-mvapich2-ohpc
\fi
\end{itemize*}
\end{tcolorbox}
\end{center}


%\vspace*{0.2cm}
\subsection{3rd Party Libraries and Tools} \label{sec:3rdparty}
\input{common/third_party_libs_intro}
\input{common/third_party_libs_petsc_centos}
\begin{center}
\begin{tcolorbox}[]
\small
\OHPC{}-provided 3rd party builds are configured to be installed
into a common top-level repository so that they can be easily exported to
desired hosts within the cluster. This common top-level path
(\path{/opt/ohpc/pub}) was previously configured to be mounted on {\em
 compute} nodes in \S\ref{sec:master_customization}, so the packages will be
immediately available for use on the cluster after installation on the {\em
 master} host.
\end{tcolorbox}
\end{center}

%\iftoggle{isCentOS}{\clearpage}
%\nottoggle{isCentOS}{\clearpage}

For convenience, \OHPC{} provides package aliases for these 3rd party libraries
and utilities that can be used to install available libraries for use with the
GNU compiler family toolchain. For parallel libraries, aliases are grouped by
MPI family toolchain so that administrators can choose a subset should they
favor a particular MPI stack.  Please refer to Appendix~\ref{appendix:manifest}
for a more detailed listing of all available packages in each of these functional
areas. To install all available package offerings within \OHPC{}, issue the
following:

% begin_ohpc_run
% ohpc_comment_header Install 3rd Party Libraries and Tools \ref{sec:3rdparty}
\begin{lstlisting}[language=bash,keywords={},upquote=true,keepspaces]
# Install 3rd party libraries/tools meta-packages built with GNU toolchain
[sms](*\#*) (*\install*) ohpc-gnu9-serial-libs
[sms](*\#*) (*\install*) ohpc-gnu9-io-libs
[sms](*\#*) (*\install*) ohpc-gnu9-python-libs
[sms](*\#*) (*\install*) ohpc-gnu9-runtimes
\end{lstlisting}
% end_ohpc_run





% begin_ohpc_run
% ohpc_comment_header Install 3rd Party Libraries and Tools \ref{sec:3rdparty}
% ohpc_command if [[ ${enable_mpi_defaults} -eq 1 ]];then
% ohpc_indent 5
\begin{lstlisting}[language=bash,keywords={},upquote=true,keepspaces]
# Install parallel lib meta-packages for all available MPI toolchains
[sms](*\#*) (*\install*) ohpc-gnu9-mpich-parallel-libs
[sms](*\#*) (*\install*) ohpc-gnu9-openmpi4-parallel-libs
\end{lstlisting}
% ohpc_indent 0
% ohpc_command fi
% end_ohpc_run


\clearpage
\section{Resource Manager Startup} \label{sec:rms_startup}
In section \S\ref{sec:basic_install}, the \SLURM{} resource manager was installed
and configured for use on both the {\em master} host and {\em compute} node
instances. With the cluster nodes up and functional, we can now startup the
resource manager services in preparation for running user jobs. Generally, this
is a two-step process that requires starting up the controller daemons on the {\em
 master} host and the client daemons on each of the {\em compute} hosts.
%Since the {\em compute} hosts were booted into an image that included the SLURM client
%components, the daemons should already be running on the {\em compute}
%hosts. 
Note that \SLURM{} leverages the use of the {\em munge} library to provide
authentication services and this daemon also needs to be running on all hosts
within the resource management pool. 
%The munge daemons should already
%be running on the {\em compute} subsystem at this point, 
The following commands can be used to startup the necessary services to support
resource management under \SLURM{}.

%\iftoggle{isCentOS}{\clearpage}

% Allow for optional sleep to wait for nodes to provision when using install
% script


% begin_ohpc_run
% ohpc_comment_header Allow for optional sleep to wait for provisioning to complete
% ohpc_command sleep ${provision_wait}
% end_ohpc_run

% begin_ohpc_run
% ohpc_comment_header Resource Manager Startup \ref{sec:rms_startup}
\begin{lstlisting}[language=bash,keywords={}]
# Start munge and slurm controller on master host
[sms](*\#*) systemctl enable munge
[sms](*\#*) systemctl enable slurmctld
[sms](*\#*) systemctl start munge
[sms](*\#*) systemctl start slurmctld

# Start slurm clients on compute hosts
[sms](*\#*) pdsh -w $compute_prefix[1-4] systemctl start munge
[sms](*\#*) pdsh -w $compute_prefix[1-4] systemctl start slurmd
\end{lstlisting}
% end_ohpc_run

%%% In the default configuration, the {\em compute} hosts will be initialized in an
%%% {\em unknown} state. To place the hosts into production such that they are
%%% eligible to schedule user jobs, issue the following:

%%% % begin_ohpc_run
%%% \begin{lstlisting}[language=bash]
%%% [sms](*\#*) scontrol update partition=normal state=idle
%%% \end{lstlisting}
%%% % end_ohpc_run



\section{Run a Test Job} \label{sec:test_job}
\input{common/slurm_test_job}

\clearpage
\appendix
{\bf \LARGE \centerline{Appendices}} \vspace*{0.2cm}

\addcontentsline{toc}{section}{Appendices}
\renewcommand{\thesubsection}{\Alph{subsection}}

\input{common/automation_appendix}
\subsection{Upgrading OpenHPC Packages}  \label{appendix:upgrade}


As newer \OHPC{} releases are made available, users are encouraged to upgrade
their locally installed packages against the latest repository versions to
obtain access to bug fixes and newer component versions. This can be
accomplished with the underlying package manager as \OHPC{} packaging maintains
versioning state across releases. Also, package builds available from the
\OHPC{} repositories have ``\texttt{-ohpc}'' appended to their names so that
wild cards can be used as a simple way to obtain updates. The following general
procedure highlights a method for upgrading existing installations.
When upgrading from a minor release older than v\OHPCVerTree{}, you will first
need to update your local \OHPC{} repository configuration to point against the
v\OHPCVerTree{} release (or update your locally hosted mirror). Refer to
\S\ref{sec:enable_repo} for more details on enabling the latest
repository. In contrast, when upgrading between micro releases on the same
branch (e.g. from v\OHPCVerTree{} to \OHPCVerTree{}.2), there is no need to
adjust local package manager configurations when using the public repository as
rolling updates are pre-configured.
 
\begin{enumerate*}
\item (Optional) Ensure repo metadata is current (on head node and in chroot
  location(s)). Package managers will naturally do this on their own over time,
  but if you are wanting to access updates immediately after a new release,
  the following can be used to sync to the latest.

\begin{lstlisting}[language=bash,keywords={}]
[sms](*\#*)  (*\clean*)
[sms](*\#*)  (*\chrootclean*)
\end{lstlisting}

\item Upgrade master (SMS) node

\begin{lstlisting}[language=bash,keywords={}]
[sms](*\#*)  (*\upgrade*) "*-ohpc"
\end{lstlisting}
  
\iftoggleverb{isWarewulf}
\begin{center}
\begin{tcolorbox}[]
\small
The version of \Warewulf{} included in \OHPC{} v1.3.6 added a new
architecture-specific package containing iPXE files. Upgraders will need to
install this package on the SMS node.
\begin{lstlisting}[language=bash,keywords={},literate={-}{-}1 {ARCH}{\arch{}}1]
[sms](*\#*)  (*\install*) warewulf-provision-server-ipxe-ARCH-ohpc
\end{lstlisting}
\end{tcolorbox}
\end{center}
\fi

\item Upgrade packages in compute image

\begin{lstlisting}[language=bash,keywords={}]
[sms](*\#*)  (*\chrootupgrade*) "*-ohpc"
\end{lstlisting}
  
\item Rebuild image(s)

\iftoggleverb{isWarewulf}
\begin{lstlisting}[language=bash,keywords={}]
[sms](*\#*) wwvnfs --chroot $CHROOT
\end{lstlisting}
\fi

\iftoggleverb{isxCAT}
\begin{lstlisting}[language=bash,keywords={},basicstyle=\fontencoding{T1}\fontsize{8.0}{10}\ttfamily,
    literate={-}{-}1 {BOSVER}{\baseos{}}1 {ARCH}{\arch{}}1]
[sms](*\#*) packimage BOSVER-x86_64-netboot-compute
\end{lstlisting}
\fi

\end{enumerate*}

\noindent In the case where packages were upgraded within the chroot compute image,
you will need to reboot the compute nodes when convenient to enable the
changes.

\subsubsection{New component variants}

As newer variants of key compiler/MPI stacks are released, \OHPC{} will
periodically add toolchains enabling the latest variant. To stay consistent
throughout the build hierarchy, minimize recompilation requirements for existing
binaries, and allow for multiple variants to coexist, unique delimiters are
used to distinguish RPM package names and module hierarchy.

In the case of a fresh install, \OHPC{} recipes default to installation of the
latest toolchains available in a given release branch. However, if upgrading a
previously installed system, administrators can {\em opt-in} to enable new
variants as they become available. To illustrate this point, consider the
previous \OHPC{} 1.3.5 release as an example which contained GCC 7.3.0
along with runtimes and libraries compiled with this toolchain.  In the case
where an admin would like to enable the newer {``gnu8''} toolchain,
installation of these additions is simplified with the use of \OHPC{}'s
meta-packages (see Table~\ref{table:groups} in Appendix
\ref{appendix:manifest}).  The following example illustrates adding the
complete ``gnu8'' toolchain.  Note that we leverage the convenience
meta-packages containing MPI-dependent builds, and we also update the
modules environment to make it the default.

\begin{lstlisting}[language=bash,keywords={}]
# Install GCC 8.x-compiled meta-packages with dependencies
[sms](*\#*)  (*\install*) ohpc-gnu8-perf-tools \
                         ohpc-gnu8-io-libs \
                         ohpc-gnu8-python-libs \
                         ohpc-gnu8-runtimes \
                         ohpc-gnu8-serial \
                         ohpc-gnu8-parallel-libs

# Update default environment
[sms](*\#*) (*\remove*) lmod-defaults-gnu7-openmpi3-ohpc
[sms](*\#*) (*\install*) lmod-defaults-gnu8-openmpi3-ohpc

\end{lstlisting}


\input{common/test_suite}
\input{common/customization_appendix_centos}
../slurm/manifest.tex
\input{common/signature}


\end{document}

